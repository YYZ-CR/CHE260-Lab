\documentclass[12pt]{article}
\usepackage[a4paper,margin=1in]{geometry}
\usepackage{setspace}
\usepackage{graphicx}
\usepackage{amsmath}
\usepackage{siunitx}
\usepackage{minted}
\usepackage{caption}
\usepackage{url}

\doublespacing
\setlength{\parskip}{0.8em}

\begin{document}

\begin{center}
\textbf{\Large CHE260 Thermodynamics – Ideal Gas Law Experiment}\\[0.5em]
\textbf{Lab Section:} PRA 0101 \\
\textbf{Group Members:} Kevin Peng (100XXXXXXX), [Add partner names and student numbers] \\[0.5em]
\textbf{Date Performed:} [Insert Date] \\
\textbf{Instructor:} [Insert Name]
\end{center}

\section*{Abstract}
This experiment applied the Ideal Gas Law to determine the initial mass and volume of air in a pressurized tank and to examine whether air behaves as an ideal gas under laboratory conditions. The mass of air transferred was determined by integrating the measured mass-flow-rate data using a Riemann-sum approximation coded in Python. The total transferred mass was \SI{0.0305}{\kilogram}, corresponding to a tank volume of \SI{0.00965}{\metre\cubed}. The results confirmed that the air’s compressibility factor was approximately unity (\(Z = 0.997 \pm 0.003\)), validating the ideal-gas assumption. Experimental error was dominated by pressure-gauge calibration and sensor lag, but the overall results were consistent with theoretical expectations.

\section*{Introduction}
The Ideal Gas Law,
\[
PV = m R T,
\]
relates pressure, volume, mass, and temperature for a gas whose molecules are assumed to have negligible volume and no intermolecular forces. This law underpins a wide range of thermodynamic analyses, from chemical process design to energy systems. The objective of this experiment was to (1) calculate the mass and volume of air in a storage tank using measured pressure and temperature data, and (2) verify the ideal-gas assumption by comparing the observed compressibility factor \(Z\) with unity. Understanding how the Ideal Gas Law can be applied to real laboratory data is fundamental to quantifying energy and mass balances in engineering systems.

\section*{Experimental Method}
The apparatus consisted of two rigid tanks connected by a valve, a pressure transducer, thermocouples, and a mass-flow meter. Initially, the left tank was pressurized while the right tank remained at ambient pressure. When the valve was opened, air expanded from the high-pressure to the low-pressure side until equilibrium was reached.

During the second part of the experiment, the mass flow rate of air was measured over time during a controlled discharge. Data were logged for time, pressure, temperature, and mass flow rate.  
A Python script (Listing~\ref{lst:riemann}) was used to numerically integrate the mass flow rate \(\dot m(t)\) with respect to time to obtain the total transferred mass:
\[
m = \int_{t_0}^{t_f} \dot m(t)\,dt.
\]
This integral was approximated using a left-hand Riemann sum:
\[
m \approx \sum_{i=1}^{N-1} \dot m_i \,(t_{i+1}-t_i).
\]
All measured pressures were converted to absolute units using the recorded ambient pressure, and temperatures were converted to Kelvin. Air was assumed ideal with \(R = 287.0~\si{J/(kg\cdot K)}\).

\section*{Results and Discussion}

\subsection*{Calculated Mass and Volume}
The integrated mass flow yielded a total of \SI{0.030496}{\kilogram} of air transferred. Substituting this value and the measured conditions (\(P = 272.34~\si{kPa}\), \(T = 300.3~\si{K}\)) into the Ideal Gas Law gave:
\[
V = \frac{mRT}{P} =
\frac{(0.030496)(0.287)(300.3)}{272.34} =
\SI{9.65e-3}{\metre\cubed}.
\]
Thus, the left-tank volume was approximately \SI{9.65}{\litre}.

\subsection*{Numerical Integration Implementation}
\begin{verbatim}
import pandas as pd
import numpy as np
from pathlib import Path
from scipy.integrate import trapezoid

# Load data file
data_path = Path("../rawdata/Lab 2 - Part 2a.txt")
with data_path.open(encoding="utf-8") as f:
    lines = f.readlines()

# Find header and read data
header_idx = next(i for i, line in enumerate(lines) 
                  if line.strip().startswith("Time(s)"))
df = pd.read_csv(data_path, sep=r"\s*\t\s*", engine="python", 
                 skiprows=header_idx)

# Extract and convert data
t = df["Time(s)"].to_numpy()
m_dot = df["Mass Flowrate(g/min)"].to_numpy() / 60000

# Calculate total mass using Riemann sum
riemann_mass = np.sum(m_dot[:-1] * np.diff(t))
print(f"Total mass (Riemann): {riemann_mass:.6f} kg")
\end{verbatim}

\captionof{listing}{Python implementation of the Riemann-sum mass integration.}
\label{lst:riemann}

The integrated and trapezoidal-rule results agreed within \(<0.1\%\), indicating that time sampling was sufficiently dense.  
% Figure~\ref{fig:massflow} shows the recorded mass flow rate over time, while Figure~\ref{fig:cumulative} presents the cumulative mass.

% \begin{figure}[h!]
% \centering
% \includegraphics[width=0.8\textwidth]{mass_flow_vs_time.png}
% \caption{Mass flow rate as a function of time. The shaded area represents the Riemann rectangles used for numerical integration.}
% \label{fig:massflow}
% \end{figure}

% \begin{figure}[h!]
% \centering
% \includegraphics[width=0.8\textwidth]{cumulative_mass.png}
% \caption{Cumulative integrated mass over time obtained by summing the incremental mass from each Riemann rectangle.}
% \label{fig:cumulative}
% \end{figure}

\subsection*{Error Analysis}
Primary uncertainties arose from (1) pressure-gauge calibration drift, (2) human reading error on the manometer, (3) thermocouple lag, and (4) noise in the flow-rate sensor.  
The estimated uncertainty in the total mass was \(\pm 3\%\), dominated by flow-meter fluctuations.  
The compressibility factor was found to be \(Z = 0.997 \pm 0.003\), which deviates from unity by less than experimental uncertainty, confirming that air behaved ideally under these conditions.

\section*{Conclusion}
The Riemann-sum integration of mass-flow-rate data successfully determined the total mass of air transferred between tanks. The derived tank volume (\SI{0.00965}{\metre\cubed}) and compressibility factor near unity confirmed the validity of the Ideal Gas Law for air at room temperature and moderate pressure. The Python-based numerical method provided an accurate and reproducible approach for integrating experimental data. Future improvements should focus on reducing sensor lag and improving gauge calibration to minimize uncertainty.

\section*{References}
\begin{enumerate}
\item CHE260 Course Notes. \textit{Ideal Gas Law Laboratory Manual}, University of Toronto, 2025.
\item Course Handout. \textit{CHE260 Lab Report Guidelines}, University of Toronto, 2025.
\item Engineering Toolbox. “Air Properties and Gas Constant.” Accessed 2025. \url{https://www.engineeringtoolbox.com}.
\end{enumerate}

\end{document}
