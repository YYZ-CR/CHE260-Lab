\documentclass[12pt,a4paper]{article}

% Packages
\usepackage[utf8]{inputenc}
\usepackage{amsmath}
\usepackage{amssymb}
\usepackage{graphicx}
\usepackage{siunitx}
\usepackage{booktabs}
\usepackage{hyperref}
\usepackage{caption}
\usepackage{float}
\usepackage[margin=1in]{geometry}

% Document information
\title{Ideal Gas Law Experiment}
\author{Your Name \\ Student ID: XXXXXXXX \\ CHE 260}
\date{\today}

\begin{document}

\maketitle

\begin{abstract}
Brief summary of the experiment, main findings, and conclusions (150-250 words).
\end{abstract}

\section{Introduction}
Background theory, objectives, and hypothesis of the experiment.

\section{Theory}
Relevant equations and theoretical background. For example, the ideal gas law:
\begin{equation}
    PV = nRT
\end{equation}
where $P$ is pressure, $V$ is volume, $n$ is number of moles, $R$ is the gas constant, and $T$ is temperature.

\section{Experimental Methods}
\subsection{Apparatus}
Description of equipment used.

\subsection{Procedure}
Step-by-step description of the experimental procedure.

\section{Results}
\subsection{Data}
Present raw data in tables using:
\begin{table}[H]
    \centering
    \caption{Experimental data}
    \begin{tabular}{@{}ccc@{}}
        \toprule
        Parameter & Value & Uncertainty \\
        \midrule
        % Add your data here
        \bottomrule
    \end{tabular}
\end{table}

\subsection{Analysis}
Calculations, graphs, and processed data.

\section{Discussion}
Interpretation of results, sources of error, and comparison with theoretical predictions.

\section{Conclusion}
Summary of findings and their significance.

\begin{thebibliography}{9}
\bibitem{ref1} 
Author Name. 
\textit{Book/Paper Title}. 
Publisher, Year.
\end{thebibliography}

\end{document}