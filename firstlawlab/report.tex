\documentclass[12pt]{article}
\usepackage[a4paper,margin=1in]{geometry}
\usepackage{setspace}
\usepackage{graphicx}
\usepackage{amsmath}
\usepackage{siunitx}
\usepackage{caption}
\usepackage{url}
\usepackage{float}

\doublespacing
\setlength{\parskip}{0.8em}

\begin{document}


\begin{center}
\textbf{\Large Verifying the First Law of Thermodynamics in a Pressurized Tank: Heat Loss, Work, and Internal Energy} \\[0.5em]
Kevin Peng (1011043238), Boya Zhang (1010855638), Yang Yang Zhang (1011437786)\\[0.5em]
CHE260 PRA 0101 \\
First Law of Thermodynamics Lab \\
October 15th, 2025 \\
\end{center}

\section*{Abstract}
The 

\section{Introduction}

\section{Experimental Method}
The apparatus consists of two rigid valve-connected tanks, pressure transducers, thermocouples, a mass flow controller, .
a pressurized tank equipped with temperature and pressure sensors, as well as a heating element to supply heat to the system (see Figure \ref{fig:apparatus}).
The software LabVIEW is utilized to record the temperature and pressure data over time. An external manometer is used to record the ambient pressure before the experiment begins. 
\begin{figure}[H]
    \centering
    \includegraphics[width=0.8\textwidth]{apparatus.png}
    \caption{Experimental apparatus setup showing the pressurized tank with temperature and pressure sensors.}
    \label{fig:apparatus}
\end{figure}


The experiment involves heating the tank to a specified temperature while monitoring the pressure and temperature changes over time.


\section{Results and Discussion}

\section{Conclusion}


\section{References}






\newpage
\begin{thebibliography}{9}

\end{thebibliography}

\end{document}
